
% Default to the notebook output style

    


% Inherit from the specified cell style.




    
\documentclass[11pt]{article}

    
    
    \usepackage[T1]{fontenc}
    % Nicer default font (+ math font) than Computer Modern for most use cases
    \usepackage{mathpazo}

    % Basic figure setup, for now with no caption control since it's done
    % automatically by Pandoc (which extracts ![](path) syntax from Markdown).
    \usepackage{graphicx}
    % We will generate all images so they have a width \maxwidth. This means
    % that they will get their normal width if they fit onto the page, but
    % are scaled down if they would overflow the margins.
    \makeatletter
    \def\maxwidth{\ifdim\Gin@nat@width>\linewidth\linewidth
    \else\Gin@nat@width\fi}
    \makeatother
    \let\Oldincludegraphics\includegraphics
    % Set max figure width to be 80% of text width, for now hardcoded.
    \renewcommand{\includegraphics}[1]{\Oldincludegraphics[width=.8\maxwidth]{#1}}
    % Ensure that by default, figures have no caption (until we provide a
    % proper Figure object with a Caption API and a way to capture that
    % in the conversion process - todo).
    \usepackage{caption}
    \DeclareCaptionLabelFormat{nolabel}{}
    \captionsetup{labelformat=nolabel}

    \usepackage{adjustbox} % Used to constrain images to a maximum size 
    \usepackage{xcolor} % Allow colors to be defined
    \usepackage{enumerate} % Needed for markdown enumerations to work
    \usepackage{geometry} % Used to adjust the document margins
    \usepackage{amsmath} % Equations
    \usepackage{amssymb} % Equations
    \usepackage{textcomp} % defines textquotesingle
    % Hack from http://tex.stackexchange.com/a/47451/13684:
    \AtBeginDocument{%
        \def\PYZsq{\textquotesingle}% Upright quotes in Pygmentized code
    }
    \usepackage{upquote} % Upright quotes for verbatim code
    \usepackage{eurosym} % defines \euro
    \usepackage[mathletters]{ucs} % Extended unicode (utf-8) support
    \usepackage[utf8x]{inputenc} % Allow utf-8 characters in the tex document
    \usepackage{fancyvrb} % verbatim replacement that allows latex
    \usepackage{grffile} % extends the file name processing of package graphics 
                         % to support a larger range 
    % The hyperref package gives us a pdf with properly built
    % internal navigation ('pdf bookmarks' for the table of contents,
    % internal cross-reference links, web links for URLs, etc.)
    \usepackage{hyperref}
    \usepackage{longtable} % longtable support required by pandoc >1.10
    \usepackage{booktabs}  % table support for pandoc > 1.12.2
    \usepackage[inline]{enumitem} % IRkernel/repr support (it uses the enumerate* environment)
    \usepackage[normalem]{ulem} % ulem is needed to support strikethroughs (\sout)
                                % normalem makes italics be italics, not underlines
    

    
    
    % Colors for the hyperref package
    \definecolor{urlcolor}{rgb}{0,.145,.698}
    \definecolor{linkcolor}{rgb}{.71,0.21,0.01}
    \definecolor{citecolor}{rgb}{.12,.54,.11}

    % ANSI colors
    \definecolor{ansi-black}{HTML}{3E424D}
    \definecolor{ansi-black-intense}{HTML}{282C36}
    \definecolor{ansi-red}{HTML}{E75C58}
    \definecolor{ansi-red-intense}{HTML}{B22B31}
    \definecolor{ansi-green}{HTML}{00A250}
    \definecolor{ansi-green-intense}{HTML}{007427}
    \definecolor{ansi-yellow}{HTML}{DDB62B}
    \definecolor{ansi-yellow-intense}{HTML}{B27D12}
    \definecolor{ansi-blue}{HTML}{208FFB}
    \definecolor{ansi-blue-intense}{HTML}{0065CA}
    \definecolor{ansi-magenta}{HTML}{D160C4}
    \definecolor{ansi-magenta-intense}{HTML}{A03196}
    \definecolor{ansi-cyan}{HTML}{60C6C8}
    \definecolor{ansi-cyan-intense}{HTML}{258F8F}
    \definecolor{ansi-white}{HTML}{C5C1B4}
    \definecolor{ansi-white-intense}{HTML}{A1A6B2}

    % commands and environments needed by pandoc snippets
    % extracted from the output of `pandoc -s`
    \providecommand{\tightlist}{%
      \setlength{\itemsep}{0pt}\setlength{\parskip}{0pt}}
    \DefineVerbatimEnvironment{Highlighting}{Verbatim}{commandchars=\\\{\}}
    % Add ',fontsize=\small' for more characters per line
    \newenvironment{Shaded}{}{}
    \newcommand{\KeywordTok}[1]{\textcolor[rgb]{0.00,0.44,0.13}{\textbf{{#1}}}}
    \newcommand{\DataTypeTok}[1]{\textcolor[rgb]{0.56,0.13,0.00}{{#1}}}
    \newcommand{\DecValTok}[1]{\textcolor[rgb]{0.25,0.63,0.44}{{#1}}}
    \newcommand{\BaseNTok}[1]{\textcolor[rgb]{0.25,0.63,0.44}{{#1}}}
    \newcommand{\FloatTok}[1]{\textcolor[rgb]{0.25,0.63,0.44}{{#1}}}
    \newcommand{\CharTok}[1]{\textcolor[rgb]{0.25,0.44,0.63}{{#1}}}
    \newcommand{\StringTok}[1]{\textcolor[rgb]{0.25,0.44,0.63}{{#1}}}
    \newcommand{\CommentTok}[1]{\textcolor[rgb]{0.38,0.63,0.69}{\textit{{#1}}}}
    \newcommand{\OtherTok}[1]{\textcolor[rgb]{0.00,0.44,0.13}{{#1}}}
    \newcommand{\AlertTok}[1]{\textcolor[rgb]{1.00,0.00,0.00}{\textbf{{#1}}}}
    \newcommand{\FunctionTok}[1]{\textcolor[rgb]{0.02,0.16,0.49}{{#1}}}
    \newcommand{\RegionMarkerTok}[1]{{#1}}
    \newcommand{\ErrorTok}[1]{\textcolor[rgb]{1.00,0.00,0.00}{\textbf{{#1}}}}
    \newcommand{\NormalTok}[1]{{#1}}
    
    % Additional commands for more recent versions of Pandoc
    \newcommand{\ConstantTok}[1]{\textcolor[rgb]{0.53,0.00,0.00}{{#1}}}
    \newcommand{\SpecialCharTok}[1]{\textcolor[rgb]{0.25,0.44,0.63}{{#1}}}
    \newcommand{\VerbatimStringTok}[1]{\textcolor[rgb]{0.25,0.44,0.63}{{#1}}}
    \newcommand{\SpecialStringTok}[1]{\textcolor[rgb]{0.73,0.40,0.53}{{#1}}}
    \newcommand{\ImportTok}[1]{{#1}}
    \newcommand{\DocumentationTok}[1]{\textcolor[rgb]{0.73,0.13,0.13}{\textit{{#1}}}}
    \newcommand{\AnnotationTok}[1]{\textcolor[rgb]{0.38,0.63,0.69}{\textbf{\textit{{#1}}}}}
    \newcommand{\CommentVarTok}[1]{\textcolor[rgb]{0.38,0.63,0.69}{\textbf{\textit{{#1}}}}}
    \newcommand{\VariableTok}[1]{\textcolor[rgb]{0.10,0.09,0.49}{{#1}}}
    \newcommand{\ControlFlowTok}[1]{\textcolor[rgb]{0.00,0.44,0.13}{\textbf{{#1}}}}
    \newcommand{\OperatorTok}[1]{\textcolor[rgb]{0.40,0.40,0.40}{{#1}}}
    \newcommand{\BuiltInTok}[1]{{#1}}
    \newcommand{\ExtensionTok}[1]{{#1}}
    \newcommand{\PreprocessorTok}[1]{\textcolor[rgb]{0.74,0.48,0.00}{{#1}}}
    \newcommand{\AttributeTok}[1]{\textcolor[rgb]{0.49,0.56,0.16}{{#1}}}
    \newcommand{\InformationTok}[1]{\textcolor[rgb]{0.38,0.63,0.69}{\textbf{\textit{{#1}}}}}
    \newcommand{\WarningTok}[1]{\textcolor[rgb]{0.38,0.63,0.69}{\textbf{\textit{{#1}}}}}
    
    
    % Define a nice break command that doesn't care if a line doesn't already
    % exist.
    \def\br{\hspace*{\fill} \\* }
    % Math Jax compatability definitions
    \def\gt{>}
    \def\lt{<}
    % Document parameters
    \title{Keras}
    
    
    

    % Pygments definitions
    
\makeatletter
\def\PY@reset{\let\PY@it=\relax \let\PY@bf=\relax%
    \let\PY@ul=\relax \let\PY@tc=\relax%
    \let\PY@bc=\relax \let\PY@ff=\relax}
\def\PY@tok#1{\csname PY@tok@#1\endcsname}
\def\PY@toks#1+{\ifx\relax#1\empty\else%
    \PY@tok{#1}\expandafter\PY@toks\fi}
\def\PY@do#1{\PY@bc{\PY@tc{\PY@ul{%
    \PY@it{\PY@bf{\PY@ff{#1}}}}}}}
\def\PY#1#2{\PY@reset\PY@toks#1+\relax+\PY@do{#2}}

\expandafter\def\csname PY@tok@cpf\endcsname{\let\PY@it=\textit\def\PY@tc##1{\textcolor[rgb]{0.25,0.50,0.50}{##1}}}
\expandafter\def\csname PY@tok@k\endcsname{\let\PY@bf=\textbf\def\PY@tc##1{\textcolor[rgb]{0.00,0.50,0.00}{##1}}}
\expandafter\def\csname PY@tok@mf\endcsname{\def\PY@tc##1{\textcolor[rgb]{0.40,0.40,0.40}{##1}}}
\expandafter\def\csname PY@tok@si\endcsname{\let\PY@bf=\textbf\def\PY@tc##1{\textcolor[rgb]{0.73,0.40,0.53}{##1}}}
\expandafter\def\csname PY@tok@nf\endcsname{\def\PY@tc##1{\textcolor[rgb]{0.00,0.00,1.00}{##1}}}
\expandafter\def\csname PY@tok@kt\endcsname{\def\PY@tc##1{\textcolor[rgb]{0.69,0.00,0.25}{##1}}}
\expandafter\def\csname PY@tok@se\endcsname{\let\PY@bf=\textbf\def\PY@tc##1{\textcolor[rgb]{0.73,0.40,0.13}{##1}}}
\expandafter\def\csname PY@tok@kn\endcsname{\let\PY@bf=\textbf\def\PY@tc##1{\textcolor[rgb]{0.00,0.50,0.00}{##1}}}
\expandafter\def\csname PY@tok@gr\endcsname{\def\PY@tc##1{\textcolor[rgb]{1.00,0.00,0.00}{##1}}}
\expandafter\def\csname PY@tok@s1\endcsname{\def\PY@tc##1{\textcolor[rgb]{0.73,0.13,0.13}{##1}}}
\expandafter\def\csname PY@tok@sr\endcsname{\def\PY@tc##1{\textcolor[rgb]{0.73,0.40,0.53}{##1}}}
\expandafter\def\csname PY@tok@mo\endcsname{\def\PY@tc##1{\textcolor[rgb]{0.40,0.40,0.40}{##1}}}
\expandafter\def\csname PY@tok@nb\endcsname{\def\PY@tc##1{\textcolor[rgb]{0.00,0.50,0.00}{##1}}}
\expandafter\def\csname PY@tok@il\endcsname{\def\PY@tc##1{\textcolor[rgb]{0.40,0.40,0.40}{##1}}}
\expandafter\def\csname PY@tok@m\endcsname{\def\PY@tc##1{\textcolor[rgb]{0.40,0.40,0.40}{##1}}}
\expandafter\def\csname PY@tok@nv\endcsname{\def\PY@tc##1{\textcolor[rgb]{0.10,0.09,0.49}{##1}}}
\expandafter\def\csname PY@tok@fm\endcsname{\def\PY@tc##1{\textcolor[rgb]{0.00,0.00,1.00}{##1}}}
\expandafter\def\csname PY@tok@ow\endcsname{\let\PY@bf=\textbf\def\PY@tc##1{\textcolor[rgb]{0.67,0.13,1.00}{##1}}}
\expandafter\def\csname PY@tok@nl\endcsname{\def\PY@tc##1{\textcolor[rgb]{0.63,0.63,0.00}{##1}}}
\expandafter\def\csname PY@tok@kd\endcsname{\let\PY@bf=\textbf\def\PY@tc##1{\textcolor[rgb]{0.00,0.50,0.00}{##1}}}
\expandafter\def\csname PY@tok@gt\endcsname{\def\PY@tc##1{\textcolor[rgb]{0.00,0.27,0.87}{##1}}}
\expandafter\def\csname PY@tok@kp\endcsname{\def\PY@tc##1{\textcolor[rgb]{0.00,0.50,0.00}{##1}}}
\expandafter\def\csname PY@tok@s2\endcsname{\def\PY@tc##1{\textcolor[rgb]{0.73,0.13,0.13}{##1}}}
\expandafter\def\csname PY@tok@vg\endcsname{\def\PY@tc##1{\textcolor[rgb]{0.10,0.09,0.49}{##1}}}
\expandafter\def\csname PY@tok@vm\endcsname{\def\PY@tc##1{\textcolor[rgb]{0.10,0.09,0.49}{##1}}}
\expandafter\def\csname PY@tok@s\endcsname{\def\PY@tc##1{\textcolor[rgb]{0.73,0.13,0.13}{##1}}}
\expandafter\def\csname PY@tok@mh\endcsname{\def\PY@tc##1{\textcolor[rgb]{0.40,0.40,0.40}{##1}}}
\expandafter\def\csname PY@tok@nc\endcsname{\let\PY@bf=\textbf\def\PY@tc##1{\textcolor[rgb]{0.00,0.00,1.00}{##1}}}
\expandafter\def\csname PY@tok@gd\endcsname{\def\PY@tc##1{\textcolor[rgb]{0.63,0.00,0.00}{##1}}}
\expandafter\def\csname PY@tok@kc\endcsname{\let\PY@bf=\textbf\def\PY@tc##1{\textcolor[rgb]{0.00,0.50,0.00}{##1}}}
\expandafter\def\csname PY@tok@go\endcsname{\def\PY@tc##1{\textcolor[rgb]{0.53,0.53,0.53}{##1}}}
\expandafter\def\csname PY@tok@gp\endcsname{\let\PY@bf=\textbf\def\PY@tc##1{\textcolor[rgb]{0.00,0.00,0.50}{##1}}}
\expandafter\def\csname PY@tok@gs\endcsname{\let\PY@bf=\textbf}
\expandafter\def\csname PY@tok@w\endcsname{\def\PY@tc##1{\textcolor[rgb]{0.73,0.73,0.73}{##1}}}
\expandafter\def\csname PY@tok@dl\endcsname{\def\PY@tc##1{\textcolor[rgb]{0.73,0.13,0.13}{##1}}}
\expandafter\def\csname PY@tok@gh\endcsname{\let\PY@bf=\textbf\def\PY@tc##1{\textcolor[rgb]{0.00,0.00,0.50}{##1}}}
\expandafter\def\csname PY@tok@sb\endcsname{\def\PY@tc##1{\textcolor[rgb]{0.73,0.13,0.13}{##1}}}
\expandafter\def\csname PY@tok@sa\endcsname{\def\PY@tc##1{\textcolor[rgb]{0.73,0.13,0.13}{##1}}}
\expandafter\def\csname PY@tok@na\endcsname{\def\PY@tc##1{\textcolor[rgb]{0.49,0.56,0.16}{##1}}}
\expandafter\def\csname PY@tok@bp\endcsname{\def\PY@tc##1{\textcolor[rgb]{0.00,0.50,0.00}{##1}}}
\expandafter\def\csname PY@tok@cm\endcsname{\let\PY@it=\textit\def\PY@tc##1{\textcolor[rgb]{0.25,0.50,0.50}{##1}}}
\expandafter\def\csname PY@tok@ge\endcsname{\let\PY@it=\textit}
\expandafter\def\csname PY@tok@nd\endcsname{\def\PY@tc##1{\textcolor[rgb]{0.67,0.13,1.00}{##1}}}
\expandafter\def\csname PY@tok@no\endcsname{\def\PY@tc##1{\textcolor[rgb]{0.53,0.00,0.00}{##1}}}
\expandafter\def\csname PY@tok@ss\endcsname{\def\PY@tc##1{\textcolor[rgb]{0.10,0.09,0.49}{##1}}}
\expandafter\def\csname PY@tok@kr\endcsname{\let\PY@bf=\textbf\def\PY@tc##1{\textcolor[rgb]{0.00,0.50,0.00}{##1}}}
\expandafter\def\csname PY@tok@gu\endcsname{\let\PY@bf=\textbf\def\PY@tc##1{\textcolor[rgb]{0.50,0.00,0.50}{##1}}}
\expandafter\def\csname PY@tok@cs\endcsname{\let\PY@it=\textit\def\PY@tc##1{\textcolor[rgb]{0.25,0.50,0.50}{##1}}}
\expandafter\def\csname PY@tok@err\endcsname{\def\PY@bc##1{\setlength{\fboxsep}{0pt}\fcolorbox[rgb]{1.00,0.00,0.00}{1,1,1}{\strut ##1}}}
\expandafter\def\csname PY@tok@cp\endcsname{\def\PY@tc##1{\textcolor[rgb]{0.74,0.48,0.00}{##1}}}
\expandafter\def\csname PY@tok@nn\endcsname{\let\PY@bf=\textbf\def\PY@tc##1{\textcolor[rgb]{0.00,0.00,1.00}{##1}}}
\expandafter\def\csname PY@tok@ne\endcsname{\let\PY@bf=\textbf\def\PY@tc##1{\textcolor[rgb]{0.82,0.25,0.23}{##1}}}
\expandafter\def\csname PY@tok@ch\endcsname{\let\PY@it=\textit\def\PY@tc##1{\textcolor[rgb]{0.25,0.50,0.50}{##1}}}
\expandafter\def\csname PY@tok@sx\endcsname{\def\PY@tc##1{\textcolor[rgb]{0.00,0.50,0.00}{##1}}}
\expandafter\def\csname PY@tok@vi\endcsname{\def\PY@tc##1{\textcolor[rgb]{0.10,0.09,0.49}{##1}}}
\expandafter\def\csname PY@tok@c1\endcsname{\let\PY@it=\textit\def\PY@tc##1{\textcolor[rgb]{0.25,0.50,0.50}{##1}}}
\expandafter\def\csname PY@tok@c\endcsname{\let\PY@it=\textit\def\PY@tc##1{\textcolor[rgb]{0.25,0.50,0.50}{##1}}}
\expandafter\def\csname PY@tok@sd\endcsname{\let\PY@it=\textit\def\PY@tc##1{\textcolor[rgb]{0.73,0.13,0.13}{##1}}}
\expandafter\def\csname PY@tok@sc\endcsname{\def\PY@tc##1{\textcolor[rgb]{0.73,0.13,0.13}{##1}}}
\expandafter\def\csname PY@tok@mi\endcsname{\def\PY@tc##1{\textcolor[rgb]{0.40,0.40,0.40}{##1}}}
\expandafter\def\csname PY@tok@vc\endcsname{\def\PY@tc##1{\textcolor[rgb]{0.10,0.09,0.49}{##1}}}
\expandafter\def\csname PY@tok@mb\endcsname{\def\PY@tc##1{\textcolor[rgb]{0.40,0.40,0.40}{##1}}}
\expandafter\def\csname PY@tok@nt\endcsname{\let\PY@bf=\textbf\def\PY@tc##1{\textcolor[rgb]{0.00,0.50,0.00}{##1}}}
\expandafter\def\csname PY@tok@gi\endcsname{\def\PY@tc##1{\textcolor[rgb]{0.00,0.63,0.00}{##1}}}
\expandafter\def\csname PY@tok@o\endcsname{\def\PY@tc##1{\textcolor[rgb]{0.40,0.40,0.40}{##1}}}
\expandafter\def\csname PY@tok@ni\endcsname{\let\PY@bf=\textbf\def\PY@tc##1{\textcolor[rgb]{0.60,0.60,0.60}{##1}}}
\expandafter\def\csname PY@tok@sh\endcsname{\def\PY@tc##1{\textcolor[rgb]{0.73,0.13,0.13}{##1}}}

\def\PYZbs{\char`\\}
\def\PYZus{\char`\_}
\def\PYZob{\char`\{}
\def\PYZcb{\char`\}}
\def\PYZca{\char`\^}
\def\PYZam{\char`\&}
\def\PYZlt{\char`\<}
\def\PYZgt{\char`\>}
\def\PYZsh{\char`\#}
\def\PYZpc{\char`\%}
\def\PYZdl{\char`\$}
\def\PYZhy{\char`\-}
\def\PYZsq{\char`\'}
\def\PYZdq{\char`\"}
\def\PYZti{\char`\~}
% for compatibility with earlier versions
\def\PYZat{@}
\def\PYZlb{[}
\def\PYZrb{]}
\makeatother


    % Exact colors from NB
    \definecolor{incolor}{rgb}{0.0, 0.0, 0.5}
    \definecolor{outcolor}{rgb}{0.545, 0.0, 0.0}



    
    % Prevent overflowing lines due to hard-to-break entities
    \sloppy 
    % Setup hyperref package
    \hypersetup{
      breaklinks=true,  % so long urls are correctly broken across lines
      colorlinks=true,
      urlcolor=urlcolor,
      linkcolor=linkcolor,
      citecolor=citecolor,
      }
    % Slightly bigger margins than the latex defaults
    
    \geometry{verbose,tmargin=1in,bmargin=1in,lmargin=1in,rmargin=1in}
    
    

    \begin{document}
    
    
    \maketitle
    
    

    
    \section{Introduction to deep
learning}\label{introduction-to-deep-learning}

    What is artificial neural network: 인공신경망(人工神經網, 영어:
artificial neural network, ANN)은 기계학습과 인지과학에서 생물학의
신경망(동물의 중추신경계중 특히 뇌)에서 영감을 얻은 통계학적 학습
알고리즘이다.

    \subsection{Interactions}\label{interactions}

\begin{itemize}
\tightlist
\item
  Neural networks account for interaction really well
\item
  Deep learning uses especially powerful neural networks(딥러닝은 nn에
  매우 강하고 좋음)

  \begin{itemize}
  \tightlist
  \item
    Text
  \item
    imge 등등
  \end{itemize}
\end{itemize}

    \subsection{Course structure}\label{course-structure}

\begin{itemize}
\tightlist
\item
  First two chapters focus on conceptual knowledge

  \begin{itemize}
  \tightlist
  \item
    Debug and tune deep learning models on conventional prediction
    problems
  \item
    Lay the foundation for progressing towrards modern applications
  \end{itemize}
\item
  This will pay off in the third and fourth chapters
\end{itemize}

    \subsection{Interactions in neural
network}\label{interactions-in-neural-network}

All network has Inpust Layer, Hidden Layer and Output Layer

\begin{figure}
\centering
\includegraphics{https://cdn-images-1.medium.com/max/479/1*QVIyc5HnGDWTNX3m-nIm9w.png}
\caption{neural network}
\end{figure}

    If layer has more hidden layers, Ability is increase (more and more)

    \section{Forward propagation}\label{forward-propagation}

    \subsection{Bank Transactions example}\label{bank-transactions-example}

    \begin{itemize}
\tightlist
\item
  Make predictions based on:

  \begin{itemize}
  \tightlist
  \item
    Number of children
  \item
    Number of exisiting accounts
  \end{itemize}
\end{itemize}

    Forwoard Propagation

\begin{longtable}[]{@{}lll@{}}
\toprule
input layer & hidden layer & output layer\tabularnewline
\midrule
\endhead
x & w & y\tabularnewline
\bottomrule
\end{longtable}

\[
xW=y
\]

    leyer image

    \subsection{Forword propagation}\label{forword-propagation}

\begin{itemize}
\tightlist
\item
  Multiply - add process
\item
  Dot product
\item
  Foward propagation for one data point at a time
\item
  Output is the prediction for that data point
\end{itemize}

    \begin{Verbatim}[commandchars=\\\{\}]
{\color{incolor}In [{\color{incolor}10}]:} \PY{k+kn}{import} \PY{n+nn}{numpy} \PY{k}{as} \PY{n+nn}{np}
         
         \PY{n}{input\PYZus{}data} \PY{o}{=} \PY{n}{np}\PY{o}{.}\PY{n}{array}\PY{p}{(}\PY{p}{[}\PY{l+m+mi}{2}\PY{p}{,} \PY{l+m+mi}{3}\PY{p}{]}\PY{p}{)} \PY{c+c1}{\PYZsh{} input value}
         
         \PY{n}{weights} \PY{o}{=} \PY{p}{\PYZob{}}\PY{l+s+s1}{\PYZsq{}}\PY{l+s+s1}{node\PYZus{}0}\PY{l+s+s1}{\PYZsq{}} \PY{p}{:} \PY{n}{np}\PY{o}{.}\PY{n}{array}\PY{p}{(}\PY{p}{[}\PY{l+m+mi}{1}\PY{p}{,} \PY{l+m+mi}{1}\PY{p}{]}\PY{p}{)}\PY{p}{,}
                   \PY{l+s+s1}{\PYZsq{}}\PY{l+s+s1}{node\PYZus{}1}\PY{l+s+s1}{\PYZsq{}} \PY{p}{:} \PY{n}{np}\PY{o}{.}\PY{n}{array}\PY{p}{(}\PY{p}{[}\PY{o}{\PYZhy{}}\PY{l+m+mi}{1}\PY{p}{,} \PY{l+m+mi}{1}\PY{p}{]}\PY{p}{)}\PY{p}{,}
                   \PY{l+s+s1}{\PYZsq{}}\PY{l+s+s1}{output}\PY{l+s+s1}{\PYZsq{}}\PY{p}{:} \PY{n}{np}\PY{o}{.}\PY{n}{array}\PY{p}{(}\PY{p}{[}\PY{l+m+mi}{2}\PY{p}{,} \PY{o}{\PYZhy{}}\PY{l+m+mi}{1}\PY{p}{]}\PY{p}{)}\PY{p}{\PYZcb{}} \PY{c+c1}{\PYZsh{} weights for input\PYZus{}data}
         
         \PY{n}{node\PYZus{}0\PYZus{}value} \PY{o}{=} \PY{p}{(}\PY{n}{input\PYZus{}data} \PY{o}{*} \PY{n}{weights}\PY{p}{[}\PY{l+s+s1}{\PYZsq{}}\PY{l+s+s1}{node\PYZus{}0}\PY{l+s+s1}{\PYZsq{}}\PY{p}{]}\PY{p}{)}\PY{o}{.}\PY{n}{sum}\PY{p}{(}\PY{p}{)} \PY{c+c1}{\PYZsh{} input 에 Weights하여 더한다.}
         \PY{n}{node\PYZus{}1\PYZus{}value} \PY{o}{=} \PY{p}{(}\PY{n}{input\PYZus{}data} \PY{o}{*} \PY{n}{weights}\PY{p}{[}\PY{l+s+s1}{\PYZsq{}}\PY{l+s+s1}{node\PYZus{}1}\PY{l+s+s1}{\PYZsq{}}\PY{p}{]}\PY{p}{)}\PY{o}{.}\PY{n}{sum}\PY{p}{(}\PY{p}{)}
         
         \PY{n}{hidden\PYZus{}layer\PYZus{}values} \PY{o}{=} \PY{n}{np}\PY{o}{.}\PY{n}{array}\PY{p}{(}\PY{p}{[}\PY{n}{node\PYZus{}0\PYZus{}value}\PY{p}{,} \PY{n}{node\PYZus{}1\PYZus{}value}\PY{p}{]}\PY{p}{)} 
         \PY{c+c1}{\PYZsh{} 각각의 input값들을 각각의 가중치와 곱한뒤 나온 값들을 더한것을 hidden\PYZus{}layer\PYZus{}values 변수에 삽입}
         
         \PY{n+nb}{print}\PY{p}{(}\PY{n}{hidden\PYZus{}layer\PYZus{}values}\PY{p}{)}
         
         \PY{n}{output} \PY{o}{=} \PY{p}{(}\PY{n}{hidden\PYZus{}layer\PYZus{}values} \PY{o}{*} \PY{n}{weights}\PY{p}{[}\PY{l+s+s1}{\PYZsq{}}\PY{l+s+s1}{output}\PY{l+s+s1}{\PYZsq{}}\PY{p}{]}\PY{p}{)}\PY{o}{.}\PY{n}{sum}\PY{p}{(}\PY{p}{)} \PY{c+c1}{\PYZsh{} hidden\PYZus{}layer\PYZus{}value와 output값을 곱했다.}
         \PY{n+nb}{print}\PY{p}{(}\PY{n}{output}\PY{p}{)}
\end{Verbatim}


    \begin{Verbatim}[commandchars=\\\{\}]
[5 1]
9

    \end{Verbatim}

    \section{Activation funtions}\label{activation-funtions}

    \subsection{Linear VS Nonlinear
Functions}\label{linear-vs-nonlinear-functions}

선형, 비선형

    \subsection{Activation functions}\label{activation-functions}

앞선 Layer 관계를 통해 확인가능

\begin{longtable}[]{@{}lll@{}}
\toprule
input & Hidden Layer & output\tabularnewline
\midrule
\endhead
2, 3 & tanh(2+3), tanh(-2+3) & 9\tabularnewline
\bottomrule
\end{longtable}

 \#\# ReLU(Rectified Linear Activation)
\href{http://mongxmongx2.tistory.com/25}{ReLU관한 링크},
\href{http://ydseo.tistory.com/41}{Gradient Vanising에 대한 문제 링크}

    ReLU를 사용하기 이전에는 activation function으로 sigmoid function을
사용했다. sigmoid function이 연속이여서 미분가능한점과 0과 1사이의 값을
가진다는 점 그리고 0에서 1로 변하는 점이 가파르기 때문에 사용해왔다.
그러나 기존에 사용하던 Simgoid fucntion을 ReLu가 대체하게 된 이유 중
가장 큰 것이 {Gradient Vanishing} 문제이다. Simgoid function은 0에서
1사이의 값을 가지는데 gradient descent를 사용해 Backpropagation 수행시
layer를 지나면서 gradient를 계속 곱하므로 gradient는 {0으로 수렴}하게
된다. 따라서 layer가 많아지면 잘 작동하지 않게 된다.

    \textbf{따라서 이러한 문제를 해결하기위해 ReLu를 새로운 activation
function을 사용한다. ReLu는 입력값이 0보다 작으면 0이고 0보다 크면
입력값 그대로를 내보낸다.}

    \begin{figure}
\centering
\includegraphics{http://cfile9.uf.tistory.com/image/246B094F57F226C0366860}
\caption{ReLU}
\end{figure}

    \section{Deeper networks}\label{deeper-networks}

    tanh(쌍곡선 함수): 수학에서 쌍곡선함수(双曲線函數)는 일반적인 삼각함수와
유사한 성질을 갖는 함수로 삼각함수가 단위원 그래프를 매개변수로 표시할
때 나오는 것처럼, 표준쌍곡선을 매개변수로 표시할 때 나온다.

tanh를 사용하는 이유: sigmod쓰지 않고 tanh를 쓰는 이유는 수식으로 쓰자면
0.5 (tanh (x)+1)=sigmoid (2x) 라 tanh (x)는 sigmoid (x)보다 두배
빨라서(추측)

그렇다면 sigmod를 왜 사용하나: sigmoid는 logistic classification에서
어디에 속하는지 분류를 하기 위해 사용했다. 일정 값을 넘어야 성공내지는
참(True)이 될 수 있기 때문에 Activation function이라고도 불렀다.

    
\includegraphics{http://cfile28.uf.tistory.com/image/253F5947579F7BC32B036E}

    ReLu를 이용하여 어떤 특징을 잡고, 이로부터 A인지 B인지 에 대해 한 구분을
한다.

    \begin{Verbatim}[commandchars=\\\{\}]
{\color{incolor}In [{\color{incolor}1}]:} \PY{k+kn}{import} \PY{n+nn}{numpy} \PY{k}{as} \PY{n+nn}{np}
        
        
        \PY{n}{input\PYZus{}data} \PY{o}{=} \PY{n}{np}\PY{o}{.}\PY{n}{array}\PY{p}{(}\PY{p}{[}\PY{o}{\PYZhy{}}\PY{l+m+mi}{1}\PY{p}{,} \PY{l+m+mi}{2}\PY{p}{]}\PY{p}{)}
        
        \PY{n}{weights} \PY{o}{=} \PY{p}{\PYZob{}}\PY{l+s+s1}{\PYZsq{}}\PY{l+s+s1}{node\PYZus{}0}\PY{l+s+s1}{\PYZsq{}} \PY{p}{:} \PY{n}{np}\PY{o}{.}\PY{n}{array}\PY{p}{(}\PY{p}{[}\PY{l+m+mi}{3}\PY{p}{,} \PY{l+m+mi}{3}\PY{p}{]}\PY{p}{)}\PY{p}{,}
                  \PY{l+s+s1}{\PYZsq{}}\PY{l+s+s1}{node\PYZus{}1}\PY{l+s+s1}{\PYZsq{}} \PY{p}{:} \PY{n}{np}\PY{o}{.}\PY{n}{array}\PY{p}{(}\PY{p}{[}\PY{l+m+mi}{1}\PY{p}{,} \PY{l+m+mi}{5}\PY{p}{]}\PY{p}{)}\PY{p}{,}
                  \PY{l+s+s1}{\PYZsq{}}\PY{l+s+s1}{output}\PY{l+s+s1}{\PYZsq{}} \PY{p}{:} \PY{n}{np}\PY{o}{.}\PY{n}{array}\PY{p}{(}\PY{p}{[}\PY{l+m+mi}{2}\PY{p}{,} \PY{o}{\PYZhy{}}\PY{l+m+mi}{1}\PY{p}{]}\PY{p}{)}\PY{p}{\PYZcb{}}
        
        \PY{n}{node\PYZus{}0\PYZus{}input} \PY{o}{=} \PY{p}{(}\PY{n}{input\PYZus{}data} \PY{o}{*} \PY{n}{weights}\PY{p}{[}\PY{l+s+s1}{\PYZsq{}}\PY{l+s+s1}{node\PYZus{}0}\PY{l+s+s1}{\PYZsq{}}\PY{p}{]}\PY{o}{.}\PY{n}{sum}\PY{p}{(}\PY{p}{)}\PY{p}{)}
        
        \PY{n}{node\PYZus{}0\PYZus{}output} \PY{o}{=} \PY{n}{np}\PY{o}{.}\PY{n}{tanh}\PY{p}{(}\PY{n}{node\PYZus{}0\PYZus{}input}\PY{p}{)}
        
        \PY{n}{node\PYZus{}1\PYZus{}input} \PY{o}{=} \PY{p}{(}\PY{n}{input\PYZus{}data} \PY{o}{*} \PY{n}{weights}\PY{p}{[}\PY{l+s+s1}{\PYZsq{}}\PY{l+s+s1}{node\PYZus{}1}\PY{l+s+s1}{\PYZsq{}}\PY{p}{]}\PY{o}{.}\PY{n}{sum}\PY{p}{(}\PY{p}{)}\PY{p}{)}
        
        \PY{n}{node\PYZus{}1\PYZus{}output} \PY{o}{=} \PY{n}{np}\PY{o}{.}\PY{n}{tanh}\PY{p}{(}\PY{n}{node\PYZus{}1\PYZus{}input}\PY{p}{)}
        
        \PY{n}{hidden\PYZus{}layer\PYZus{}outputs} \PY{o}{=} \PY{n}{np}\PY{o}{.}\PY{n}{array}\PY{p}{(}\PY{p}{[}\PY{n}{node\PYZus{}0\PYZus{}output}\PY{p}{,} \PY{n}{node\PYZus{}1\PYZus{}output}\PY{p}{]}\PY{p}{)}
        
        \PY{n}{output} \PY{o}{=} \PY{p}{(}\PY{n}{hidden\PYZus{}layer\PYZus{}outputs} \PY{o}{*} \PY{n}{weights}\PY{p}{[}\PY{l+s+s1}{\PYZsq{}}\PY{l+s+s1}{output}\PY{l+s+s1}{\PYZsq{}}\PY{p}{]}\PY{p}{)}\PY{o}{.}\PY{n}{sum}\PY{p}{(}\PY{p}{)}
        
        \PY{n+nb}{print}\PY{p}{(}\PY{n}{output}\PY{p}{)}
\end{Verbatim}


    \begin{Verbatim}[commandchars=\\\{\}]
-5.99995084645

    \end{Verbatim}

    \begin{Verbatim}[commandchars=\\\{\}]
{\color{incolor}In [{\color{incolor}23}]:} \PY{k+kn}{import} \PY{n+nn}{numpy} \PY{k}{as} \PY{n+nn}{np}
         
         \PY{n}{input\PYZus{}data} \PY{o}{=} \PY{n}{np}\PY{o}{.}\PY{n}{array}\PY{p}{(}\PY{p}{[}\PY{l+m+mi}{3}\PY{p}{,} \PY{l+m+mi}{5}\PY{p}{]}\PY{p}{)}
         
         \PY{k}{def} \PY{n+nf}{relu}\PY{p}{(}\PY{n+nb}{input}\PY{p}{)}\PY{p}{:}
             \PY{n}{relu\PYZus{}value} \PY{o}{=} \PY{n+nb}{max}\PY{p}{(}\PY{l+m+mi}{0}\PY{p}{,} \PY{n+nb}{input}\PY{p}{)}
             \PY{k}{return} \PY{n}{relu\PYZus{}value}
         
         \PY{n}{weights} \PY{o}{=} \PY{p}{\PYZob{}}\PY{l+s+s1}{\PYZsq{}}\PY{l+s+s1}{node\PYZus{}0}\PY{l+s+s1}{\PYZsq{}}\PY{p}{:}\PY{n}{np}\PY{o}{.}\PY{n}{array}\PY{p}{(}\PY{p}{[}\PY{l+m+mi}{2}\PY{p}{,} \PY{l+m+mi}{4}\PY{p}{]}\PY{p}{)}\PY{p}{,}
                   \PY{l+s+s1}{\PYZsq{}}\PY{l+s+s1}{node\PYZus{}1}\PY{l+s+s1}{\PYZsq{}}\PY{p}{:} \PY{n}{np}\PY{o}{.}\PY{n}{array}\PY{p}{(}\PY{p}{[}\PY{l+m+mi}{4}\PY{p}{,} \PY{o}{\PYZhy{}}\PY{l+m+mi}{5}\PY{p}{]}\PY{p}{)}\PY{p}{,}
                   \PY{l+s+s1}{\PYZsq{}}\PY{l+s+s1}{node\PYZus{}0\PYZus{}output}\PY{l+s+s1}{\PYZsq{}} \PY{p}{:} \PY{n}{np}\PY{o}{.}\PY{n}{array}\PY{p}{(}\PY{p}{[}\PY{o}{\PYZhy{}}\PY{l+m+mi}{1}\PY{p}{,} \PY{l+m+mi}{2}\PY{p}{]}\PY{p}{)}\PY{p}{,}
                   \PY{l+s+s1}{\PYZsq{}}\PY{l+s+s1}{node\PYZus{}1\PYZus{}output}\PY{l+s+s1}{\PYZsq{}} \PY{p}{:} \PY{n}{np}\PY{o}{.}\PY{n}{array}\PY{p}{(}\PY{p}{[}\PY{l+m+mi}{2}\PY{p}{,} \PY{l+m+mi}{1}\PY{p}{]}\PY{p}{)}\PY{p}{,}\PY{c+c1}{\PYZsh{} 왜 ([1, 2])는 안되는가?}
                   \PY{l+s+s1}{\PYZsq{}}\PY{l+s+s1}{output}\PY{l+s+s1}{\PYZsq{}} \PY{p}{:} \PY{n}{np}\PY{o}{.}\PY{n}{array}\PY{p}{(}\PY{p}{[}\PY{o}{\PYZhy{}}\PY{l+m+mi}{3}\PY{p}{,} \PY{l+m+mi}{7}\PY{p}{]}\PY{p}{)}\PY{p}{\PYZcb{}}
         
         \PY{n}{node\PYZus{}0\PYZus{}input} \PY{o}{=} \PY{p}{(}\PY{n}{input\PYZus{}data} \PY{o}{*} \PY{n}{weights}\PY{p}{[}\PY{l+s+s1}{\PYZsq{}}\PY{l+s+s1}{node\PYZus{}0}\PY{l+s+s1}{\PYZsq{}}\PY{p}{]}\PY{p}{)}\PY{o}{.}\PY{n}{sum}\PY{p}{(}\PY{p}{)}
         \PY{n}{node\PYZus{}0\PYZus{}output} \PY{o}{=} \PY{n}{relu}\PY{p}{(}\PY{n}{node\PYZus{}0\PYZus{}input}\PY{p}{)}
         \PY{n+nb}{print}\PY{p}{(}\PY{n}{node\PYZus{}0\PYZus{}output}\PY{p}{)}
         \PY{n}{node\PYZus{}1\PYZus{}input} \PY{o}{=} \PY{p}{(}\PY{n}{input\PYZus{}data} \PY{o}{*} \PY{n}{weights}\PY{p}{[}\PY{l+s+s1}{\PYZsq{}}\PY{l+s+s1}{node\PYZus{}1}\PY{l+s+s1}{\PYZsq{}}\PY{p}{]}\PY{p}{)}\PY{o}{.}\PY{n}{sum}\PY{p}{(}\PY{p}{)}
         \PY{n}{node\PYZus{}1\PYZus{}output} \PY{o}{=} \PY{n}{relu}\PY{p}{(}\PY{n}{node\PYZus{}1\PYZus{}input}\PY{p}{)}
         \PY{c+c1}{\PYZsh{} print(node\PYZus{}1\PYZus{}output)}
         \PY{n}{hidden\PYZus{}layer\PYZus{}outputs} \PY{o}{=} \PY{n}{np}\PY{o}{.}\PY{n}{array}\PY{p}{(}\PY{p}{[}\PY{n}{node\PYZus{}0\PYZus{}output}\PY{p}{,} \PY{n}{node\PYZus{}1\PYZus{}output}\PY{p}{]}\PY{p}{)}
         
         \PY{n}{node\PYZus{}0\PYZus{}output2} \PY{o}{=} \PY{p}{(}\PY{n}{hidden\PYZus{}layer\PYZus{}outputs} \PY{o}{*} \PY{n}{weights}\PY{p}{[}\PY{l+s+s1}{\PYZsq{}}\PY{l+s+s1}{node\PYZus{}0\PYZus{}output}\PY{l+s+s1}{\PYZsq{}}\PY{p}{]}\PY{p}{)}\PY{o}{.}\PY{n}{sum}\PY{p}{(}\PY{p}{)}
         \PY{n}{node\PYZus{}0\PYZus{}output3} \PY{o}{=} \PY{n}{relu}\PY{p}{(}\PY{n}{node\PYZus{}0\PYZus{}output2}\PY{p}{)}
         \PY{c+c1}{\PYZsh{} print(node\PYZus{}0\PYZus{}output3)}
         
         \PY{n}{node\PYZus{}1\PYZus{}output2} \PY{o}{=} \PY{p}{(}\PY{n}{hidden\PYZus{}layer\PYZus{}outputs} \PY{o}{*} \PY{n}{weights}\PY{p}{[}\PY{l+s+s1}{\PYZsq{}}\PY{l+s+s1}{node\PYZus{}1\PYZus{}output}\PY{l+s+s1}{\PYZsq{}}\PY{p}{]}\PY{p}{)}\PY{o}{.}\PY{n}{sum}\PY{p}{(}\PY{p}{)}
         \PY{n}{node\PYZus{}1\PYZus{}output3} \PY{o}{=} \PY{n}{relu}\PY{p}{(}\PY{n}{node\PYZus{}1\PYZus{}output2}\PY{p}{)}
         \PY{n+nb}{print}\PY{p}{(}\PY{n}{node\PYZus{}1\PYZus{}output3}\PY{p}{)}
         
         \PY{n}{hidden\PYZus{}layer\PYZus{}outputs2} \PY{o}{=} \PY{n}{np}\PY{o}{.}\PY{n}{array}\PY{p}{(}\PY{p}{[}\PY{n}{node\PYZus{}0\PYZus{}output3}\PY{p}{,} \PY{n}{node\PYZus{}1\PYZus{}output3}\PY{p}{]}\PY{p}{)}
         
         \PY{n}{output} \PY{o}{=} \PY{p}{(}\PY{n}{hidden\PYZus{}layer\PYZus{}outputs2} \PY{o}{*} \PY{n}{weights}\PY{p}{[}\PY{l+s+s1}{\PYZsq{}}\PY{l+s+s1}{output}\PY{l+s+s1}{\PYZsq{}}\PY{p}{]}\PY{p}{)}\PY{o}{.}\PY{n}{sum}\PY{p}{(}\PY{p}{)}
         
         \PY{n+nb}{print}\PY{p}{(}\PY{n}{output}\PY{p}{)}
\end{Verbatim}


    \begin{Verbatim}[commandchars=\\\{\}]
26
52
364

    \end{Verbatim}

    26*7

    \begin{Verbatim}[commandchars=\\\{\}]
{\color{incolor}In [{\color{incolor}22}]:} \PY{l+m+mi}{26}\PY{o}{*}\PY{l+m+mi}{7}
\end{Verbatim}


\begin{Verbatim}[commandchars=\\\{\}]
{\color{outcolor}Out[{\color{outcolor}22}]:} 182
\end{Verbatim}
            
    \subsection{Representation
Learning(표현학습)}\label{representation-learninguxd45cuxd604uxd559uxc2b5}

    \href{http://hugman.re.kr/blog/repr_learning}{Representation}

\begin{figure}
\centering
\includegraphics{http://hugman.re.kr/static/media/uploads/apple_rep.png}
\caption{표현학습의 예}
\end{figure}

    숫자 덩어리(Tensor)의 장점 1. 도메인의 전문가 없이도 쌓아놓은 데이터만
가지고도 정보를 구축할 수 있습니다. 2. 어떠한 유형의 데이터든 하나로
묶을 수 있습니다. 3. 사람의 기호로 표현하기 힘든 그 무엇(Feeling, Sense
등등 에 대한것을 표현?)

    이런식의 Tensor들이 모여 컴퓨터 스스로 하나의 특징표현을 만들어내는것 =
Representation Learning(표현학습)

    미리 공부하면 좋은 학습들 - 퍼셉트론 - 하강경사법 - Overfitting -
오류역전파 알고리즘 - 소벨마스크 - 선형회귀/로지스틱 회귀 - Sigmoid

    \section{The need for optimization}\label{the-need-for-optimization}

    \subsection{A baseline neural Network}\label{a-baseline-neural-network}

    오차를 줄이는방법(시작한다는 말)

    \section{Gradient descent}\label{gradient-descent}

    오차를 줄이기 위해 기울기(\textbf{미분})를 이용하여 점진적으로 오차를
줄여가는 방식

    \href{http://aikorea.org/cs231n/optimization-1}{gradient}

    \section{Backpropagation}\label{backpropagation}

    \href{http://llnntms.tistory.com/31}{back propagation(역전파)}

    역전파 알고리즘을 적용시키기 이전에 MLP에 대해 몇가지 알아야할 것들이
있다.

\begin{enumerate}
\def\labelenumi{\alph{enumi}.}
\item
  초기 가중치, weight값은 랜덤으로 주어진다.
\item
  각각 노드, node는 하나의 퍼셉트론으로 생각한다. 즉, 노드를 지나칠
  때마다 활성함수를 적용한다. 활성함수를 적용하기 이전을 net, 이후를
  out이라고 하겠다. 다음 레이어의 계산에는 out값을 사용한다. 마지막
  out이 output이 된다.
\item
  활성함수는 시그모이드 sigmoid 함수로 한다.(시그모이드 함수에 관해서는
  https://en.wikipedia.org/wiki/Sigmoid\_function)

  (미분하기 용이해서 대표적으로 쓰이는 함수)
\end{enumerate}

    \section{Classification 모델}\label{classification-uxbaa8uxb378}

    카테고리에 해당하는 것을 나누는것 대표적으로 onehot encoding
\href{http://www.openwith.net/?p=617}{link}

    \section{Modelvalidation}\label{modelvalidation}

    \href{http://daryan.tistory.com/21}{교차검증 혹은 유효검사}

\begin{enumerate}
\def\labelenumi{\arabic{enumi}.}
\item
  데이터 50\% : 50\% 분할
\item
  Leave one out cross validation (LOOCV)
\item
  \textbf{k-fold}
  \includegraphics{http://cfile7.uf.tistory.com/image/24153F4F5883D16E021099}
  정해진 것은 없지만, 보통 10개를 하고, 필요에 따라서 더 많이 합니다.
  명심해야할 것은 K가 적어질수록 모델의 평가는 편중될수 밖에 없으며,
  결과도 정확하지 않습니다. k가 높을수록 평가는 bias가 낮아지지만,
  결과의 분산이 높을수 있습니다. (즉 k=100일 경우 정확도가
  1\textasciitilde{}100점이라고 나올수 있으며, 이럴경우에는 50점의
  정확도를 가진다고 표현할수 있으며, k=20일 경우 정확도가
  25\textasciitilde{}75점이라고 나와서 , 정확도가 50이라고 표현할수도
  있습니다. 동일한 정확도이지만 k-100일 경우가 분산이 더 높은 결과임은
  틀림없습니다.)
\end{enumerate}

    \section{Overfitting and Underfitting(Model
capacity)}\label{overfitting-and-underfittingmodel-capacity}


    % Add a bibliography block to the postdoc
    
    
    
    \end{document}
